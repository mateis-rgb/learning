\documentclass[french]{article}
	\usepackage[T1]{fontenc}
	\usepackage{lmodern}        % Caractères plus lisibles
	\usepackage{babel}          % Réglages linguistiques (avec french)
	\pagestyle{empty}
% \usepackage{charter} % Use the Charter font

\usepackage[
	a4paper, % Paper size
	top=1in, % Top margin
	bottom=1in, % Bottom margin
	left=1in, % Left margin
	right=1in, % Right margin
	% showframe % Uncomment to show frames around the margins for debugging purposes
]{geometry}

\setlength{\parindent}{1em} % Paragraph indentation
\setlength{\parskip}{2em} % Vertical space between paragraphs

% \usepackage[ddmmyyyy]{datetime}

\usepackage{fancyhdr} % Required for customizing headers and footers

\fancypagestyle{firstpage}{%
	\fancyhf{} % Clear default headers/footers
	\renewcommand{\headrulewidth}{0pt} % No header rule
	\renewcommand{\footrulewidth}{1pt} % Footer rule thickness
}

\fancypagestyle{subsequentpages}{%
	\fancyhf{} % Clear default headers/footers
	\renewcommand{\headrulewidth}{1pt} % Header rule thickness
	\renewcommand{\footrulewidth}{1pt} % Footer rule thickness
}

\AtBeginDocument{\thispagestyle{firstpage}} % Use the first page headers/footers style on the first page
\pagestyle{subsequentpages} % Use the subsequent pages headers/footers style on subsequent pages

%----------------------------------------------------------------------------------------

\begin{document}

Voici deux questions de QCM sur les complexes et la trigonométrie pour un niveau BUT1 GEII : \\

Ces questions couvrent les concepts de la forme trigonométrique des complexes et des relations entre les complexes et les fonctions trigonométriques. \\

\textbf{1. Question 1 :}
   	Soit \( z = 1 + i \), où \( i \) est l’unité imaginaire. Quelle est la forme trigonométrique de \( z \) ?
	
	\begin{itemize}
		\item[1.] \( z = \sqrt{2} \left( \cos \frac{\pi}{4} + i \sin \frac{\pi}{4} \right) \)
		\item[2.] \( z = \sqrt{3} \left( \cos \frac{\pi}{6} + i \sin \frac{\pi}{6} \right) \)
		\item[3.] \( z = 2 \left( \cos \frac{\pi}{2} + i \sin \frac{\pi}{2} \right) \)
		\item[4.] \( z = \frac{1}{2} \left( \cos \frac{\pi}{3} + i \sin \frac{\pi}{3} \right) \)
	\end{itemize}

\textbf{2. Question 2 :}
	Soit \( z = e^{i \theta} \) avec \( \theta = \frac{\pi}{3} \). Quelle est la partie réelle de \( z \) ?
	
	\begin{itemize}
		\item[1.] \( \frac{1}{2} \)
		\item[2.] \( \frac{\sqrt{2}}{2} \)
		\item[3.] \( 1 \)
		\item[4.] \( \frac{\sqrt{3}}{2} \)
   	\end{itemize}
\end{document}