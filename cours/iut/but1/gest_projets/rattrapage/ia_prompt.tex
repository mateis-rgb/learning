\documentclass[titlepage]{article}

\usepackage[french]{babel}
\usepackage[utf8]{inputenc}
\usepackage[T1]{fontenc}
\usepackage{markdown}

\title{Document de prompt et réponse de l'IA générative}
\author{Matéis RAGON}

\begin{document}
	\begin{titlepage}
		\maketitle
	\end{titlepage}

	\tableofcontents

	\clearpage

	\section{Introduction}

	Tout d'abord, il est important de noter que l'utilisation de l'IA dans le cadre de ce sujet de rattrapage est purement informative. Elle ne servira en aucun cas à fournir ou étayer des réponses à des questions. Ainsi, toute coïncidence textuelle entre une réponse générée par l'IA et le texte rendu sera fortuite. \\

	Ensuite, la structure du document ci-dessous sera de la forme suivante : chaque couple prompt/réponse sera mis ensemble. Le numéro du prompt indique la séparation entre chaque prompt. Le prompt en question est indiqué en dessous de son numéro et la réponse de l'IA est sous forme de sous-sections nommées et numérotées.  

	\section{$1^{er}$ prompt}
	
	> J'ai compris le document, cependant, dis-moi ce que tu comprends du sujet de rattrapage, n'omets aucun détail.
	
	\subsection{Proposition d'une méthode d'amélioration continue}
	La méthode d'amélioration continue promue par William Edwards Deming est le cycle PDCA (Plan-Do-Check-Act). Ce cycle permet d'optimiser les processus de gestion de projet à travers une boucle itérative, chaque étape apportant des ajustements pour une amélioration continue du projet.

	Les quatre étapes du cycle PDCA sont :
	\begin{itemize}
		\item \textbf{Plan} : Identifier les objectifs, les exigences et les solutions possibles.
		\item \textbf{Do} : Mettre en œuvre la solution selon le plan établi.
		\item \textbf{Check} : Vérifier et évaluer les résultats par rapport aux objectifs.
		\item \textbf{Act} : Ajuster les actions en fonction des résultats de la phase de vérification.
	\end{itemize}

	\subsection{Exigences métiers}
	Les exigences métiers sont liées à l'objectif principal du projet et à la satisfaction des besoins du client. Pour la station météorologique, voici trois exigences métiers possibles :
	\begin{itemize}
		\item \textbf{Fiabilité des données} : La station doit fournir des données précises et fiables sur la température, l'humidité et la force du vent.
		\item \textbf{Accessibilité des données} : Les utilisateurs doivent pouvoir accéder aux données en temps réel via une interface en ligne.
		\item \textbf{Sécurisation des données} : Les données collectées doivent être transmises et stockées de manière sécurisée.
	\end{itemize}
	Ces exigences se situent dans la phase de \textbf{spécification des exigences métiers} du projet.

	\subsection{Exigences fonctionnelles}
	Les exigences fonctionnelles précisent les fonctionnalités spécifiques du projet. Pour la station météorologique, on pourrait envisager les exigences suivantes :
	\begin{itemize}
		\item \textbf{Mesure en temps réel} : La station doit mesurer la température, l'humidité et la force du vent en temps réel.
		\item \textbf{Transmission horaire des données} : Les données collectées doivent être envoyées toutes les heures à un serveur distant sécurisé.
		\item \textbf{Affichage des données} : Les données doivent être affichées en temps réel sur une carte interactive.
	\end{itemize}
	Ces exigences fonctionnelles correspondent à la phase de \textbf{spécification des exigences fonctionnelles} du projet.

	\subsection{Exigences techniques}
	Les exigences techniques concernent la mise en œuvre technologique du projet. Pour la station météorologique, voici trois exigences techniques :
	\begin{itemize}
		\item \textbf{Précision des capteurs} : Les capteurs de température, d'humidité et de force du vent doivent avoir une précision suffisante pour garantir des données fiables.
		\item \textbf{Connectivité réseau} : La station doit être capable de se connecter au réseau pour transmettre les données de manière sécurisée toutes les heures.
		\item \textbf{Sécurisation des données} : Le stockage et la transmission des données doivent respecter des protocoles de sécurité robustes pour protéger les informations collectées.
	\end{itemize}
	Ces exigences techniques sont spécifiées dans la phase de \textbf{spécification des exigences techniques} du projet.

	\subsection{Grandes tâches de réalisation}
	Les grandes tâches de réalisation représentent les actions principales à entreprendre pour mener à bien le projet. Voici trois grandes tâches :
	\begin{itemize}
		\item \textbf{Installation des capteurs} : Installer les capteurs nécessaires pour mesurer la température, l'humidité et la force du vent sur la Tour Eiffel.
		\item \textbf{Développement de la plateforme de transmission} : Développer une plateforme permettant la collecte et la transmission des données à un serveur sécurisé toutes les heures.
		\item \textbf{Mise en place du système d'affichage en temps réel} : Créer une carte interactive pour afficher les données météorologiques en temps réel.
	\end{itemize}
	Ces tâches sont réalisées dans la phase de \textbf{réalisation de la solution} du projet.

	\subsection{Préparation à la livraison}
	Avant de livrer la solution, il est essentiel de réaliser les vérifications suivantes :
	\begin{itemize}
		\item \textbf{Tests de performance} : S'assurer que la station fonctionne correctement en conditions réelles (mesures précises, transmission fiable des données).
		\item \textbf{Tests de sécurité} : Vérifier que les données sont sécurisées pendant la transmission et le stockage.
		\item \textbf{Validation des exigences fonctionnelles et techniques} : Vérifier que la solution respecte bien toutes les exigences métiers, fonctionnelles et techniques spécifiées.
	\end{itemize}
	Ces actions sont réalisées dans les phases de \textbf{tests et validation}, avant la \textbf{livraison finale} du projet.

	\section{$2^e$ prompt}

	> Dans le sujet que je t'ai fourni, qu'est-ce qui est sous-entendu par exigence métier ?

	\subsection{Exigences Métiers}
	Dans le contexte de ce projet, une \textbf{exigence métier} fait référence aux besoins ou attentes spécifiques du \textbf{client} ou des \textbf{utilisateurs} concernant les \textbf{résultats} et \textbf{objectifs} du projet. Ces exigences sont directement liées à l' \textbf{utilité} et à la \textbf{fonctionnalité} de la solution mise en place, dans ce cas, la station météorologique installée sur la Tour Eiffel.

	Les exigences métiers se concentrent sur \textbf{ce que le projet doit accomplir} pour répondre aux objectifs du client ou de l'utilisateur final, sans entrer dans les détails techniques.

	\subsubsection{Exemples d'Exigences Métiers}

	\paragraph{Fiabilité des données}
	La station météorologique doit garantir des mesures précises et constantes de la température, de l'humidité et de la force du vent. Cela est crucial pour les utilisateurs qui dépendent de ces données pour diverses applications (prévisions météorologiques, analyse climatique, etc.).

	\paragraph{Accessibilité des données}
	Les données doivent être accessibles aux utilisateurs (scientifiques, public, autorités, etc.) via une interface simple, comme une carte interactive ou une plateforme en ligne. C'est une exigence métier car elle répond à la manière dont les utilisateurs souhaitent interagir avec la station météorologique.

	\paragraph{Sécurisation des données}
	Les informations collectées doivent être protégées contre tout accès non autorisé, car elles peuvent être sensibles. Par exemple, des utilisateurs peuvent vouloir que les données soient sécurisées pour éviter des manipulations ou des fuites d'informations.

	\subsubsection{Résumé}
	Les exigences métiers concernent \textbf{ce que le client souhaite obtenir} à partir de la station météorologique, telles que la fiabilité, l'accessibilité et la sécurité des données. Ces besoins sont \textbf{fonctionnels} et orientés vers la \textbf{satisfaction du client ou de l'utilisateur final}, et non sur \textbf{comment} ces besoins seront réalisés techniquement.

	\clearpage

	\section{Sources et conception}

	Le modèle de LLM utilisé est $GPT4-o$. \\

	Ce document a été fait en LaTeX, écrit sous Visual Studio Code et compilé avec $pdflatex$.

\end{document}
