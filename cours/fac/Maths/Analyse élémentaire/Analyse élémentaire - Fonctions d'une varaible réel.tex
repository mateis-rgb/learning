\documentclass{article}

\usepackage[utf8]{inputenc}   
\usepackage[T1]{inputenc}
\usepackage{amsfonts}
          
\author{Matéis RAGON}
\title{Analyse Élémentaire - Fonction d'une variable réel}
\date{Sept. 2023}
 
\begin{document}
	\maketitle
	
	\section*{Définition :} 	
		\begin{enumerate}
  			\item Une fonction d'une variable réelle est la donnée d'un domaine $X \subset \mathbb{R}$ d'un ensemble donné $Y \subset \mathbb{R}$ et pour tout $x \in \mathbb{R}$, d'un unique $y\in\mathbb{R}$ appelé l'image de $x$ par la fonction $f$.
			\item L'image de $f$ on dit $f(x)$ est l'ensemble de toutes les images de $f(x)=\{ f(x), x\in\ X \}$.
			\item Un antécédent de $y \in Y$ par la fonction $f$ est un élément de $x\in X$ tel que $y=f(x)$.
			\item L'ensemble des antécédents de $y$ par la fonction $f$ est noté $f^{-1}(\{ y \})=\{ x \in X, f(x)=y \}$
		\end{enumerate}
	
	\section*{Exemple :}
		$f[ 1; 3] \mapsto \mathbb{R}$ \\
		$f: x\mapsto 2x+1$

		L'image de $f: f([1;3[) \subset [3; 7[$ \\
		Soit $y\in[3; 7[$, il existe un antécédent dans $[1;3[$
		$3 \leqslant y < 7$ \\
		$\iff 2 \leqslant y-1 < 6$ \\
		$\iff 1 \leqslant \frac{y-1}{2} < 3$
		Donc ici $\frac{y-1}{2} = x$.
		Donc $x$ est un antécédent de $y$ dans $[1;3[$ donc $[3;7[ \subset f([1;3[)$.
		L'image de $f$ est $[3;7[$
		- Les antécédents de 4 $(\in [3;7[)$
		On cherche les $x\in [1;3[$ tels que :
		$$
		f(x)=4 \\
		\iff 2x+1=4 \\
		\iff 2x=3 \\
		\iff x=\frac3 2
		$$
		De plus $\frac 3 2 \in [1;3[$, donc 4 a un antécédent par la fonction $f$ qui est $\frac3 2$.

	\section*{Définition des fonctions usuels :}
		\begin{enumerate}
			\item Polynômes :
				Soit $n \in \mathbb{N}$ et soient $a_{0}, ..., a_{n} \in \mathbb{R}$
				La fonction $f: \mathbb{R} \mapsto \mathbb{R}$,
				$f:x \mapsto a_{0} + a_{1}x+...+a_{n}x^{n}$
				est une fonction polynôme de degré $n$.

			\item Valeur absolue : \\
				$\mid\cdot\mid \mathbb{R}\mapsto\mathbb{R}$ \\
				$x \mapsto \left \{
					\begin{array}{r l}
						-x & si & x<0 \\
						x & sinon
					\end{array}
				\right .$
				\begin{enumerate}
					\item Propriété (inégalité triangulaire)
					Soient $x,y \in\mathbb{R}$, on a :
					$\mid x+y\mid \leqslant\mid x\mid + \mid y\mid$
					$\mid \overrightarrow{X}+\overrightarrow{Y}\mid \leqslant\mid \overrightarrow{X}\mid + \mid \overrightarrow{Y}\mid$
			
					\item Preuve :
					Soient $x, y\in \mathbb{R}$,
					$-\mid x\mid \leqslant x\leqslant\mid x\mid$
					et $-\mid y\mid \leqslant y\leqslant\mid y\mid$
					En somme, les inégalitées donnent : $-(\mid x\mid + \mid y\mid) \leqslant x+y\leqslant\mid x\mid+\mid y\mid$
					\item A SAVOIR!
					$-K\leqslant t\leqslant K$
					$\iff \mid t\mid\leqslant K$
				\end{enumerate}
			Donc : $\iff\mid x+y\mid \leqslant\mid x\mid + \mid y\mid$

		\item Racine Carré:
			$\sqrt\cdot: [0; +\infty[\rightarrow\mathbb{R}$
			$x \mapsto\sqrt x$
			avec $\sqrt{x^{2}} = \mid x\mid$ et $\sqrt{x}^{2} = x$
		\item Exponentielle
		\item Logarithme néperien
		\item $sin$, $cos$, ...
		
		\end{enumerate}

	Definition: Graph d'une fonction
	On appelle le graph d'une fonction $X\subset\mathbb{R}\rightarrow Y\subset\mathbb{R}$ le sous-ensemble de $\mathbb{R}\times\mathbb{R}$ forme des couples $\{x, f(x)\}\in X, Y$

	Définition: Injéctivité - Surjéctivité
	Une fonction $f: X \rightarrow Y$ est dite :
	- Injective si:
	$\forall x_{1}, x_{2} \in X$, $f(x_{1}) = f(x_{2})$
	$\iff$ $x_{1} = x_{2}$

	- Surjective si :
	$\forall y\in Y, \exists x\in X$ tel que $f(x) = y$

	- Bijective  si elle est injective **et** surjective. 
\end{document}