\documentclass{article}
\usepackage{amsmath, amssymb}
\usepackage[utf8]{inputenc}
\usepackage[T1]{fontenc}

\title{Algèbre élémentaire - Équations trigonométriques}
\author{Matéis R.}
\date{Septembre 2023}

\begin{document}
\maketitle

On souhaite résoudre des équations du type :

\begin{equation*}
\begin{aligned}
I - \begin{cases}
\sin(x) &= a & & x\in \mathbb{R} \\
\cos(x) &= a & & a \in \left[ -1, 1 \right[
\end{cases}
\end{aligned}
\end{equation*}

Ou :

\begin{equation*}
\begin{aligned}
II - \begin{cases}
\sin(x) &= \sin(y) & \text{pour} & x, y \in \mathbb{R} \\
\cos(x) &= \cos(y) &
\end{cases}
\end{aligned}
\end{equation*}

\section{Proposition (Résolution des équations $I$)}

Soit $a \in \left[-1, 1 \right]$

L'équation $\sin(x) = a$ d'inconnue $x$ a toujours une et une seule solution dans l'intervalle $\left[-\frac{\pi}{2}, \frac{\pi}{2}\right]$ (notée $\arcsin(a)$ en analyse). L'ensemble des solutions de $\sin(x) = a$ est l'ensemble $\{ \arcsin(a) + 2k\pi, k \in \mathbb{Z} \} \cup \{ -\arcsin(a) + 2k\pi, k \in \mathbb{Z} \}$

L'équation $\cos(x) = a$ d'inconnue $x$ a toujours une et une seule solution dans l'intervalle $\left[0, \pi \right]$ (notée $\arccos(a)$ en analyse). L'ensemble des solutions de $\cos(x) = a$ est l'ensemble $\{ \arccos(a) + 2k\pi, k \in \mathbb{Z} \} \cup \{ -\arccos(a) + 2k\pi, k \in \mathbb{Z} \}$

Exemple : 
$\sin(x) = -\frac{\sqrt{3}}{2}$ 
$\iff$ $x \in \{ \frac{\pi}{3} + 2k\pi, k \in \mathbb{Z} \} \cup \{\frac{2\pi}{3} + 2k\pi, k \in \mathbb{Z} \}$
$\iff$ $x=\frac{\pi}{3} [ 2\pi ]$ ou $x=\frac{2\pi}{3} [ 2\pi ]$ avec la notation modulo

$\cos(x) = -\frac{\sqrt{3}}{2}$ 
$\iff$ $x \in \{ \frac{\pi}{6} + 2k\pi, k \in \mathbb{Z} \} \cup \{-\frac{\pi}{6} + 2k\pi, k \in \mathbb{Z} \}$
$\iff$ $x=\frac{\pi}{6} [ 2\pi ]$ ou $x=-\frac{\pi}{6} [ 2\pi ]$ avec la notation modulo

\section{Proposition (Résolution des équations $II$)}

Soit $y \in \mathbb{R}$

L'ensemble des solutions de l'équation $\sin(x)=\sin(y)$ est $\{ y + 2k\pi, k \in \mathbb{Z} \} \cup \{ -y + 2k\pi, k \in \mathbb{Z} \}$

L'ensemble des solutions de l'équation $\cos(x)=\cos(y)$ est $\{ y + 2k\pi, k \in \mathbb{Z} \} \cup \{ -y + 2k\pi, k \in \mathbb{Z} \}$

Exemple :  $\sin(2x) = \sin(x)$
$\iff$ $2x=x+2k\pi, k \in \mathbb{Z}$
	ou $2x=\pi-x+2k\pi, k \in \mathbb{Z}$
$\iff$ $x=2k\pi, k \in \mathbb{Z}$
	ou $3x=\pi+2k\pi, k \in \mathbb{Z}$
$\iff$ $x=2k\pi, k \in \mathbb{Z}$
	ou $x=\frac{\pi}{3} + \frac{2k\pi}{3}, k \in \mathbb{Z}$

Avec la notation modulo 
$\iff$ $x = 0 [2\pi]$
	ou $3x = \pi [2\pi]$
$\iff$ $x = 0 [2\pi]$
	ou $x = \frac{\pi}{3} [2\pi]$

\section{Proposition (Résolution des équations avec $\tan()$)}

Soit $a \in \mathbb{R}$

L'équation $\tan(x)=a$ possède une et une seule solution notée $\arctan(a)$ dans l'intervalle $\left]-\frac{\pi}{2}, \frac{\pi}{2} \right[$. L'ensemble des solutions de l'équation est $\{ \arctan(a) + k\pi, k \in \mathbb{Z} \}$ ou $\{ \arctan(a)[\pi] \}$ en notation modulo.

\section{Formules d'addition et de duplication :}

Théorème : soient $\alpha, \beta \in \mathbb{R}$

Alors : $\sin(\alpha + \beta) = \sin(\alpha)\cos(\beta) + \sin(\beta)\cos(\alpha)$

Corollaire :
\begin{align*}
\sin(\alpha + \beta) &= \sin(\alpha)\cos(\beta) + \sin(\beta)\cos(\alpha) \\
\sin(\alpha - \beta) &= \sin(\alpha)\cos(\beta) - \sin(\beta)\cos(\alpha) \\
\cos(\alpha + \beta) &= \cos(\alpha)\cos(\beta) - \sin(\alpha)\sin(\beta) \\
\cos(\alpha - \beta) &= \cos(\alpha)\cos(\beta) + \sin(\alpha)\sin(\beta) \\
\tan(\alpha+\beta) &= \frac{\tan(\alpha) + \tan(\beta)}{1 - \tan(\alpha)\tan(\beta)}
\end{align*}

\end{document}